\section*{はじめに}

小学校5,6年生と中学校1年生に1人1台の学習者用端末を優先的に整備する経費 2318億円を盛り込んだ令和元年度補正予算が、令和2年1月30日に参院本会議で可決されました。2023年度までに全国の小中学校に1人1大宇の学習者用コンピューターを国主導で整備する「\textbf{GIGAスクール構想}」が名実ともに始動したのです。

マイクロソフトは「\textbf{地球上のすべての児童生徒、学生のの学習効果の向上を支援します。}」といった理念のもと、文部科学省が発表した「GIGAスクール構想」に対応する新しい教育機関向けソリューションとして「\textbf{GIGAスクールパッケージ}」を2020年2月4日よりパートナー企業と連携して提供を開始しました。

マイクロソフトは、これまで教育機関向けのクラウドソリューションや PC などをパートナー企業と連携して提供するとともに、ICT を活用した授業を円滑に行うための研修をこの1年間だけでも3万人以上の日本の教員の皆様に実施するなど、子供たちが21世紀の国際競争社会を生き抜いていくための力 ``\textbf{Fugure-ready skill}'' (フューチャーレディースキル)をテーマに、「\textbf{子供の学び方}」、「\textbf{先生の教え方}」、「\textbf{学校での働き方}」を変革を支援してきました。そうした中で、今回の文部科学省「GIGAスクール構想」の1日も早い実現に貢献させていただくために、マイクロソフトが提唱する「GIGAスクールパッケージ」を提供いたします。

「GIGAスクールパッケージ」は、大きく5つで構成されています。

\begin{itemize}
    \item GIGAスクール対応PC
    \item GIGAスクールに向けた教育プラットフォーム
    \item 大規模な端末展開とアカウント管理手法の提供
    \item 学びと働き方を同時に改革する教員研修の無償提供
    \item 「教育情報セキュリティポリシーに関するガイドライン」準拠
\end{itemize}

本書では「\textbf{大規模端末展開とアカウント管理手法}」に関して詳細に説明していきます。1人1大の学習者用コンピューターを導入する小・中学校がよりスムーズな導入と運用、そして新たな教育実現に向けたリファレンスになることを切に願っています。

\section*{本書の構成と読み方}

本書は以下のシナリオに沿って構成されています。

\begin{enumerate}
    \item Office 365 の導入、運用のためのシナリオ
    \item Windows PC の大規模展開を行うためのシナリオ
    %\item Windows PC の運用管理を行うためのシナリオ
    %\item Windows PC と Office 365 を使った授業のシナリオ
\end{enumerate}

本書ではこれらのシナリオに沿いながら、実際にどのようにシステム設計をおこない、構築、運用していくかといったところまでを解説いたします。